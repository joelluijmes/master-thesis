\clearpage
\section{Research Methodology}

The final solution for the company is a model to assess road quality based on different sources of data. Methodologically we can see this as a design problem. In which the research goal is to design, develop and evaluate an artefact (solution)\cite{Hevner2004}. In order to ensure a certain degree of scientific rigor, a design science methodology is used \cite{Versloot2019}.

In this respect Hevner et al. \cite{Hevner2004} is commonly referred to, in which researches perform cyclical process of development and evaluation. However, Sein et al. \cite{Sein2011} state that traditional design research has drawbacks in which scientific rigor is more valued than organizational value. Thereby failing to recognize that the artefact actually emerges from interaction with the organization. In their paper, the authors propose a new method called Action Design Research (ADR). This method based on a more iterative approach. This approach reflects the premise that IT artefacts are shaped by the organization, both during development and use. It conceptualizes research process as the interwoven activities of: building the IT artefact, intervening in the organization and evaluating it concurrently \cite{Sein2011}.

ADR consists of the following four stages \cite{Sein2011}.
\begin{enumerate}
\item \textbf{Problem Formulation}: identifies and conceptualizes a research opportunity. The trigger is a problem perceived or anticipated by the researchers. Input can come from practitioners, end-users, researchers, existing technology and/or review of prior research.
\item \textbf{Building, Intervention, and Evaluation}: iterative process which interweaves building the IT artefact, intervention in the organization, and evaluation (BIE). The problem framing of stage one is used to generate an initial design of the IT artefact which is further shaped by organizational use and subsequent design cycles. During BIE, the problem and artefact are continually evaluated.
\item \textbf{Reflection and Learning}: recognizes that the research problem involves more than simply solving a problem. Conscious reflection on the problem framing, theories chosen, and the emerging ensemble is critical to ensure contributions to knowledge are identified. This is a continuous stage which run in parallel with the first two stages.
\item \textbf{Formalization of Learning}: has the goal to generalize the learning. Making the learning from specific-and-unique to a generic-and-abstract. 
\end{enumerate}

This thesis proposal states the problem formulation that road quality assessment is a costly and labor-intensive procedure (see section \ref{section:problem}). On approval of this proposal we'll move to the following stages. \textit{Building, Intervention, and Evaluation} is the experimental process in which the activities from above are performed. This stage will be done in accordance with the organization. With the purpose to make artefacts in such way that they are usable by the organization. The following stage \textit{Reflection and Learning} is done in accordance with the university of which the contributions and learnings are written down in my thesis. Finally the stage \textit{Formalization of Learning} will be done in the discussion and reflection of my thesis.
